%! Author = Barasingha
%! Date = 26/02/20

\usepackage[utf8]{inputenc}
\usepackage[T1]{fontenc}

%\usepackage{showframe} % TEMPORARY : useful for debugging layouts

%\areaset[BCOR]{width}{height}
\areaset[0mm]{200mm}{110mm} % Set the type area Default for a5paper is 210mm/148mm
%\usepackage[landscape]{geometry} % Landscape view
%\usepackage[width=154truemm,height=216truemm,center]{crop} % resizing margins

\usepackage{comment} % adds {comment} environment

%\usepackage{ifthen} % Conditional blocks

% BEGIN GUITAR CHORDS
% Doc available here : (http://tug.ctan.org/info/latex4musicians/latex4musicians.pdf)
% (http://mirrors.ibiblio.org/CTAN/macros/latex/contrib/guitarchordschemes/guitarchordschemes_en.pdf)
\usepackage{guitarchordschemes}

% General chords parameters
\setchordscheme{
  x-unit=8pt, % chord size, x (before rotation)
  y-unit=8pt, % chord size, y (before rotation)
  rotate=270,
  finger-format=\footnotesize, % fingering font
  position-format=\footnotesize, % position number font
  name-format=\bfseries, % chord name font
  name-below=true,
  name-distance=.01em,
  strings=6,
  finger-radius={.3175},
  ringing-style={circle, minimum width=16pt, draw, inner sep=0pt},
  muted-style={cross out, draw, minimum size=5pt, inner sep=0pt, outer sep=0pt},
  tuning={,,,,,},
  restrict-bounding-box=false
}
% END GUITAR CHORDS

% BEGIN SONGS PACKAGE
% For actual song texts. see (https://ctan.org/pkg/songs)
\usepackage[chorded,showmeasures]{songs} % For guitar version
%\usepackage[chorded,nomeasures,noscripture,noshading]{songs} % Intermediate version
%\usepackage[lyric,nomeasures,noscripture,noshading]{songs} % For singing only version

% Song numbering
\settowidth\songnumwidth{\printsongnum{999}} % Width of the shaded box
%\nosongnumbers

% Verse numbering
%\renewcommand{\theversenum}{\Roman{versenum}}
\renewcommand{\printversenum}[1]{\itshape\normalsize#1.}
\settowidth\versenumwidth{\printversenum{99\kern1mm}}
%\noversenumbers

% Song appearance
%\renewcommand{\lyricfont}{\large} % Font style for lyrics
\renewcommand{\notefont}{\slshape}

%\renewcommand{\notebgcolor}{red} % Notes background color
%\renewcommand{\snmbgcolor}{red} % Song n° color

\renewcommand{\printchord}{\itshape\bfseries} % Font style for chords names

\renewcommand{\sharpsymbol}{\ensuremath{^\sharp}}

\afterpreludeskip=-10pt % Space after prelude
\beforepostludeskip=-8pt % Space before postlude

\renewcommand{\clineparams}{ % spacing between chords and lyrics
  \baselineskip=9pt
  \lineskiplimit=0.5pt
  \lineskip=0.5pt
}

\renewcommand{\extendprelude}{
  \showrefs\showauthors
  {\bfseries\songcopyright\par}
}
\renewcommand{\extendpostlude}{
  \songlicense\unskip
}

%\renewcommand{\lastcolglue}{0pt} % To make the last column to be flush with the bottom of the page

% The default level is 3, which avoids column-breaks, page-breaks, and page-turnswithin songs whenever possible.
% (Page-turns are page-breaks after odd-numberedpages in two-sided documents, or after all pages in one-sided documents.)
% Level 2 avoids page-breaks and page-turns but allows column-breaks within songs.
% Level 1 avoids only page-turns within songs. Level 0 turns off the song-positioning algorithm entirely.
% This causes songs to be positioned wherever TEX thinks is best based on penalty settings (see \vvpenalty and \spenalty).
\songpos{2}

\renewcommand{\songchapter}{\chapter}
\renewcommand{\songsection}{\section}

\renewcommand{\capo}[1]{\iftranscapos\transpose{#1}\else\musicnote{Capo #1}\fi} % Command for "Capo X" blocs

% Font kerning correction
% The following code fixes a corner case where '/', '(', and ')' characters in a superscripted chord name would overstrike the lyrics below them.
% Just feeling fancy here, not gonna lie
\begin{comment}

\newcommand{\smraise}[1]{\hbox{\small#1}}
%\newcommand{\smraise}[1]{\raise2pt\hbox{\small#1}}
\newcommand{\myslash}{\smraise/}
\newcommand{\myopenparen}{\smraise(}
\newcommand{\mycloseparen}{\smraise)}

%\renewcommand{\chordlocals}{
%  \catcode‘(=\active
%  \catcode‘)=\active
%  \catcode‘/=\active}
\global\let(\myopenparen
\global\let)\mycloseparen
\global\let/\myslash

%{\chordlocals}
\end{comment}

% END SONGS PACKAGE