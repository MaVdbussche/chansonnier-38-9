%! Author = Barasingha
%! Date = 26/02/20

\usepackage[utf8]{inputenc}
\usepackage[T1]{fontenc}

\usepackage[landscape]{geometry} % Landscape view

\usepackage{comment} % adds {comment} environment

% BEGIN GUITAR CHORDS
% Doc available here : (http://tug.ctan.org/info/latex4musicians/latex4musicians.pdf)
% (http://mirrors.ibiblio.org/CTAN/macros/latex/contrib/guitarchordschemes/guitarchordschemes_en.pdf)
\usepackage{guitarchordschemes}

% general chords parameters
\setchordscheme{
    x-unit=8pt, % chord size, x (before rotation)
    y-unit=8pt, % chord size, y (before rotation)
    rotate=270,
    finger-format=\footnotesize, % fingering font
    position-format=\footnotesize, % position number font
    name-format=\bfseries, % chord name font
    name-below=true,
    name-distance=.01em,
    strings=6,
    finger-radius={.3175},
    ringing-style={circle, minimum width=16pt, draw, inner sep=0pt},
    muted-style={cross out, draw, minimum size=5pt, inner sep=0pt, outer sep=0pt},
    tuning={,,,,,},
    restrict-bounding-box=false
}
% END GUITAR CHORDS

\usepackage{ifpdf}
% For actual song texts. see (https://ctan.org/pkg/songs)
\usepackage[chorded,showmeasures,noshading]{songs} % use [lyric] instead of [chorded] to omit all chords in output file